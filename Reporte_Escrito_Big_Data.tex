% !TeX root = Reporte_Escrito_Big_Data.tex


\documentclass[12pt]{article}
\usepackage[utf8]{inputenc}
\usepackage[spanish]{babel}
\usepackage{amsmath, amssymb}
\usepackage{listings}
\usepackage{xcolor}
\usepackage{geometry}
\usepackage{hyperref}


% Configuración de márgenes
\geometry{a4paper, margin=1in}

% Configuración de bloques de código
\lstset{
    backgroundcolor=\color{gray!10},
    basicstyle=\ttfamily\footnotesize,
    frame=single,
    breaklines=true,
    keywordstyle=\color{blue},
    commentstyle=\color{green!50!black},
    stringstyle=\color{red}
}

\begin{document}

\begin{center}
    {\Large \textbf{Universidad de La Sabana}}\\
    \vspace{0.2cm}
    {\large Maestría en Analítica Aplicada}\\
    \vspace{3cm}
    {\LARGE \textbf{Taller 2: Contenerización con Docker y PostgreSQL}}\\
    \vspace{3cm}
    Esteban Bernal - Informatics Engineering - 271938\\
    Juan Montes - Informatics Engineering - 272113\\
    Juan Gómez - Informatics Engineering - 286774\\
    \vfill
    Profesor: Hugo Franco, Ph.D. \\
    Herramientas de Big Data \\
    \vspace{1cm}
\end{center}

\newpage

\section{Problema}
El objetivo de este taller fue contenerizar una base de datos \textbf{PostgreSQL} usando \textbf{Docker}, cargar datos de población, esperanza de vida y PIB per cápita, y realizar consultas y comparaciones de rendimiento entre inserción fila por fila (INSERT) y carga masiva (COPY).

\section{Método de solución}
La solución se abordó mediante:
\begin{itemize}
    \item Creación de un contenedor Docker con PostgreSQL mediante \texttt{docker-compose}.
    \item Desarrollo de un script en Python para cargar y transformar los datos.
    \item Inserción de datos en la base usando dos métodos distintos: INSERT y COPY.
    \item Ejecución de consultas SQL de análisis comparativo.
\end{itemize}

\subsection{Algoritmo propuesto}
\begin{lstlisting}[language=Python, caption=Fragmento del código en Python]
conn = psycopg2.connect(
    host="localhost",
    port=5433,
    database="bigdatatools1",
    user="psqluser",
    password="psqlpass"
)
\end{lstlisting}

\section{Resultados}
Se obtuvieron los siguientes tiempos de ejecución:
\begin{itemize}
    \item \textbf{INSERT fila por fila:} 15.7 segundos
    \item \textbf{COPY en bloque:} 0.17 segundos
\end{itemize}

La diferencia muestra que COPY es considerablemente más eficiente para cargas masivas.

\section{Discusión}
Los resultados permiten concluir:
\begin{itemize}
    \item COPY es el método recomendado para carga inicial de grandes volúmenes.
    \item El uso de Docker facilita la portabilidad del entorno de base de datos.
    \item PostgreSQL combinado con Python y Pandas permite integrar análisis avanzados de datos.
\end{itemize}

\section*{Referencias}
\begin{itemize}
    \item Documentación oficial de PostgreSQL: \url{https://www.postgresql.org/docs/}
    \item Psycopg2: \url{https://www.psycopg.org/}
    \item Pandas: \url{https://pandas.pydata.org/}
\end{itemize}

\end{document}
